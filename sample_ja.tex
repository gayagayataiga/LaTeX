%!TEX program = lualatex
\documentclass[a4paper,12pt]{ltjsarticle}
\usepackage{graphicx}
\usepackage{amsmath}

\title{GitHub CodespacesでのLaTeXサンプル文書}
\author{LaTeX狂いの豊工生}
\date{\today}

\begin{document}

\maketitle

\section{はじめに}
この環境はレポートをたくさん書くであろう豊工生のために作成されました。
環境を作成したのは、LaTeXを先輩に進められて触ってみたいけど、環境構築がうまくいかないという後輩の声を聞いたからです。
GitHub Codespacesを使うことで、環境構築の手間を省き、すぐにLaTeX文書の作成を始めることができます。

主な操作方法はREADME.mdを参照してください。

このような環境はこのリポジトリだけでなく、他の方がも作成して公開しています。
それらのリポジトリも参考にしてみてください。
私は、自分好みに環境を構築したかったので、自分で生成AIを使いながら作成しました。

これを作成するにあたって、学校においてあるLaTeXの本を参考にしました。
本を読むことで、レポートで使わなかったような機能を使えたので、私は満足です。

\section{数式}
数式の例:
\begin{equation}
    E = mc^2
\end{equation}

インライン数式も使用できます:$a^2 + b^2 = c^2$

\section{リスト}
\subsection{箇条書き}
\begin{itemize}
    \item 項目1
    \item 項目2
    \item 項目3
\end{itemize}

\subsection{番号付きリスト}
\begin{enumerate}
    \item 項目1
    \item 項目2
    \item 項目3
\end{enumerate}

\section{まとめ}
この環境でGitHub Codespaces上でLaTeX文書を作成できます!

\end{document}
